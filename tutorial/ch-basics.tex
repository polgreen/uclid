\chapter{Basics: Types and Statements}
\begin{uclidlisting}[htbp]
    \lstinputlisting[language=uclid,style=uclidstyle]{../examples/tutorial/ex2.1-alu.ucl}
    \caption{Model of a simple ALU}
\label{ex:alu}
\end{uclidlisting}

This chapter will provide an overview of \uclid{}'s type system and modelling features. Let us start with Example~\ref{ex:alu}, a model of a simple arithmetic logic unit (ALU). 

\section{Types in \uclid{}}

Types supported by \uclid{} are of the following kinds: 
\begin{enumerate}
    \item \keyword{integer}: the type of mathematical integers.
    \item \keyword{boolean}: the Boolean type. This type has two values: \keyword{true} and \keyword{false}.
    \item \keyword{bv\textit{W}}: The family of bit-vector types parameterized by their width ($W$).
    \item \keyword{enum}: enumerated types.
    \item Tuples and records.
    \item Array types.
    \item Uninterpreted types.
\end{enumerate}

An enumerated type is used in line~3 of Example~\ref{ex:alu}. This declares a \textit{type synonym}: \codelike{cmd_t} is an alias for the enumerated type consisting of three values: \codelike{add}, \codelike{sub} and \codelike{mov_imm}. The input \codelike{cmd} is then declared to be of type \codelike{cmd_t} on line~8. 

The input \codelike{valid} is of type \keyword{boolean}. Register indices \codelike{r1} and \codelike{r2} are bit-vectors of width 3 (\keyword{bv3}), while \codelike{immed}, \codelike{result}, \codelike{r1val} and \codelike{r2val} are bit-vectors of width 8 (\keyword{bv8}).

Line~3 declares a type synonym for a \keyword{record}. It declares \codelike{result_t} as consisting of two fields: a Boolean field \codelike{valid} and a bit-vector field \codelike{value}. The output \codelike{result} is declared to be of type \codelike{result_t} on Line~9.

The final point of interest, type-wise, is line~10. The state variable \codelike{regs} is declared to be of type array: indices to the array are of type \codelike{bv3} and elements of the array are of type \codelike{bv8}. This is used to model an 8-entry register file, where each register is a bit-vector of width 8.

\section{Statements in \uclid{}}

\uclid{} allows the modeling of both sequential and parallel computation. Sequential computation is performed by defining a \keyword{procedure} while parallel computation occurs in the \keyword{init} and \keyword{next} blocks. 

\subsection{Parallel vs. Sequential Computation}
Assignments inside procedures are called \textbf{sequential assignments} and must be of the form \codelike{variable = expression;} Assignments inside the \codelike{init} and \codelike{next} blocks are \textbf{parallel assignments} must be of the form \codelike{variable' = expression;} and \uclid{} uses the notation \codelike{variable'} to distinguish parallel from sequential assignments.

An example of sequential computation is the following:

%\begin{uclidlisting}[htbp]
\begin{lstlisting}[language=uclid,style=uclidstyle]
  x = 1;
  x = x + 2;
  x = x + 3;
\end{lstlisting}

Recall that these sequential assignments \emph{must} appear inside a procedure. Executing these three statements will result \codelike{x} having the value 6. 

    In contrast, the following sequence of parallel assignments is \textbf{not} allowed and will result in a compiler error.
\begin{lstlisting}[language=uclid,style=uclidstyle]
  x' = 1;
  x' = x + 2;
  x' = x + 3;
\end{lstlisting}
Only a single parallel assignment to a state/output variable is allowed in a code block. And since parallel assignments are computed in data-flow order, the order in which they are specified does not matter. This means that the following two snippets of code are equivalent:

\begin{lstlisting}[language=uclid,style=uclidstyle]
  x' = 0;
  y' = x' + 1;
\end{lstlisting}

\begin{lstlisting}[language=uclid,style=uclidstyle]
  y' = x' + 1;
  x' = 0;
\end{lstlisting}

\uclid{} will determine that since \codelike{y'} depends on the value of \codelike{x'}, \codelike{x'} has to computed first and this value has to be used in the computation of \codelike{y'}. This is regardless of the order in which these assignments appear in the \codelike{init}/\codelike{next} block.
%\caption{Example of sequential assignment.}
%\label{ex:seq-assignment}
%\end{uclidlisting}

    %\lstinputlisting
\subsection{Procedures}
Example~\ref{ex:alu} demonstrates how sequential computation is used in concert with parallel computation. The \codelike{procedure} \codelike{set\_init\_state} (lines~13--18) is used to initialize the values of the registers (state variable \codelike{regs}), the variables \codelike{cnt} and \codelike{result}. Since this procedure updates the module's state variables, a \codelike{modifies} clause (line~14) is required to explicitly specify that these updates are intended. The procedure is called on line~21 in the \codelike{init} block. Updates to a state variable not mentioned in a modifies clause will result in a compilation error.

Similarly, \codelike{procedure} \codelike{exec\_cmd} executes a single ALU command and returns (line 25) a single value of type \codelike{result\_t}. The procedure is invoked on line~41 in the \codelike{next} block, and its return value is assigned to the output variable \codelike{result}. Note we are again using the notation \codelike{result'} to refer to parallel assignment.

\subsection{For Loops}
The procedure \codelike{set\_init\_state} uses a \keyword{for} loop to initialize each value in the array \codelike{regs} to the bit-vector value 1.\footnote{\codelike{1bv8} here refers to the bit-vector value 1 of width 8.} The loop iterates over the values between 0 and 7 (both-inclusive). 

The range over which a \keyword{for} loop iterates must be defined by two numeric literals.

\subsection{If and Case Statements}
Also worth pointing out are the \keyword{if} statement that appears on line~28, and the \keyword{case} statement that appears on line~31. Syntax for \keyword{if} statements should be familiar. 

\keyword{case} statements are delimited by \keyword{case} and \keyword{esac} and contain within them a list of boolean expressions and associated statement blocks. These expressions are evaluated in the order in which appear, and if any of them evaluate to \codelike{true}, the corresponding block is executed. If none of the case-expressions evaluate to \codelike{true}, nothing is executed. The keyword \keyword{default} can be used as a ``catch-all'' case like in C/C++. 

\subsection{Expressions}

The syntax for expressions in \uclid{} is similar to languages like C/C++/Java. Index \codelike{i} of array \codelike{regs} is accessed using the syntax \codelike{regs[i]}. Field \codelike{value} in the record \codelike{result} is accessed as \codelike{result.value}. 

\section{Verification Model}
This section briefly describes the execution semantics of Example~\ref{ex:alu}.

\subsection{Initialization}
Execution of the model in Example~\ref{ex:alu} starts with the \keyword{init} block. This block invokes \codelike{set\_init\_state} and assigns initial values to \codelike{regs}, \codelike{cnt} and \codelike{result.value}. The other variables (e.g. \codelike{r1val} and \codelike{r2val}) are not assigned to in the \keyword{init} block and will be initialized non-deterministically.

\subsection{Next State Computation}
The next state of each state variable in the model is computed according to the \keyword{next} block. Any variables not assigned to in the \keyword{next} block retain their ``old'' values. 

The \keyword{input} variables of the model are assigned (possibly different) non-deterministic values for each step of the transition system. These values can be controlled by using assumptions. Indeed, the model uses the three assumptions on lines 45--46 to constrain the input to the ALU to always be an \codelike{add} operation, where both operands refer to register index 0. 

\subsection{Verification}
As in Example~\ref{ex:fib-model}, the verification script in Example~\ref{ex:alu} unrolls the transition system for 5 steps and checks if the \keyword{invariant} on line~47 is violated in any of these steps. 

\subsection{Running \uclid{}}

Running \uclid{} on Example~\ref{ex:alu} produces the following output.

\begin{Verbatim}[frame=single, samepage=true]
$ uclid examples/tutorial/ex2.1-alu.ucl 
Successfully parsed 1 and instantiated 1 module(s).
6 assertions passed.
0 assertions failed.
0 assertions indeterminate.
Finished execution for module: main.
\end{Verbatim}

\uclid{} is able to prove that the \keyword{invariant} on line~46 holds for all states reachable within 5 steps of the initial state, under the assumptions specified in lines~45--46.
