\chapter{Compositional Reasoning with Modules}
This chapter describes \uclid{}'s features for compositional and modular verification.

We will use a running example of a CPU model constructed in the \uclid{} and use bounded model checking to prove that the execution of this CPU is deterministic: i.e. we show that given two identical instruction memories, the state updates performed by this CPU will be identical.

\section{Common Type Definitions Across Modules}
Example~\ref{ex:cpu-common} shows a module that defines only type synonyms. Such a module can be used to share type definitions across other modules. The types declared in Example~\ref{ex:cpu-common} are \emph{imported} in lines~3-7 of module \codelike{cpu} declared in Example~\ref{ex:cpu-cpu-p1}.

\label{sec:cpu-model}
\begin{uclidlisting}[htbp]
\begin{lstlisting}[language=uclid,style=uclidstyle]
// This module declares types that are used in the rest of the model.
module common {
  // addresses are uninterpreted types.
  type addr_t = bv8;
  type word_t = bv8;
  // memory
  type mem_t = [addr_t]word_t;
  // CPU operation.
  type op_t   = enum { op_mov, op_add, op_sub, op_load, op_store };
}
\end{lstlisting}
\caption{Module \codelike{common} of the CPU model}
\label{ex:cpu-common}
\end{uclidlisting}

\begin{uclidlisting}[htbp]
\begin{lstlisting}[language=uclid,style=uclidstyle]
module cpu {
  // Import type aliases from common.
  type addr_t = common :: addr_t;
  type mem_t  = common :: mem_t;
  type word_t = common :: word_t;
  type op_t   = common :: op_t;
  
  // type of register file.
  type regindex_t; 
  type regs_t = [regindex_t]word_t;

  // imem is the program memory; an input to the module.
  input imem : mem_t;
  
  var dmem : mem_t;
  var regs : regs_t;
  var pc   : addr_t;

  // These are uninterpreted functions.
  function inst2op   (i : word_t) : op_t;
  function inst2reg0 (i : word_t) : regindex_t;
  function inst2reg1 (i : word_t) : regindex_t;
  function inst2imm  (i : word_t) : word_t;
  function inst2addr (i : word_t) : addr_t;

  var inst   : word_t;
  var result : word_t;

  procedure exec_inst(inst : word_t, pc : addr_t)
    returns (result : word_t, pc_next : addr_t)
    modifies regs;
  {
    var op     : op_t;
    var r0ind  : regindex_t;
    var r1ind  : regindex_t;
    var r0     : word_t;
    var r1     : word_t;

    op = inst2op(inst);
    r0ind = inst2reg0(inst);
    r1ind = inst2reg1(inst);
    r0 = regs[r0ind];
    r1 = regs[r1ind];
    case
      (op == op_mov)     : { result = inst2imm(inst); }
      (op == op_add)     : { result = r0 + r1; }
      (op == op_sub)     : { result = r0 - r1; }
      (op == op_load)    : { result = dmem[inst2addr(inst)]; }
      (op == op_store)   : { result = r0; }
    esac
    pc_next = pc + 1bv8;
    regs[r0ind] = result;
  }
  // continued in Example 4.3.
\end{lstlisting}
\caption{Variable and Procedure Declarations of the \codelike{cpu} module}
\label{ex:cpu-cpu-p1}
\end{uclidlisting}

\begin{uclidlisting}[htbp]
\begin{lstlisting}[language=uclid,style=uclidstyle]
  // continued from Example 4.2.

  // Define initial state for the modules.
  init {
    assume (forall (r : regindex_t) :: regs[r] == 0bv8);
    assume (forall (a : addr_t) :: dmem[a] == 0bv8);
    pc = 0bv8;
    inst = 0bv8;
  }

  next {
    // get the next instruction.
    inst = imem[pc];
    // execute it
    call (result, pc) = exec_inst(inst, pc);
  }
}
\end{lstlisting}
\caption{\keywordbf{init} and \keywordbf{next} blocks of the \codelike{cpu} module}
\label{ex:cpu-cpu-p2}
\end{uclidlisting}

\begin{uclidlisting}[htbp]
\begin{lstlisting}[language=uclid,style=uclidstyle]
module main {

  // Import types
  type addr_t     = common :: addr_t;
  type mem_t      = common :: mem_t;
  type word_t     = common :: word_t;
  type op_t       = common :: op_t;
  type regindex_t = cpu :: regindex_t;
  
  // instruction memory is the same for both CPUs.
  var imem : mem_t;

  // Create two instances of the CPU module.
  instance cpu_i_1 : cpu(imem : (imem));
  instance cpu_i_2 : cpu(imem : (imem));

  init {
  }

  next {
    // Invoke CPU 1 and CPU 2.
    next (cpu_i_1);
    next (cpu_i_2);
  }
  
  // These are our properties.
  invariant eq_regs : (forall (ri : regindex_t) :: 
                        cpu_i_1->regs[ri] == cpu_i_2->regs[ri]);
  invariant eq_mem  : (forall (a : addr_t) :: 
                        cpu_i_1->dmem[a] == cpu_i_2->dmem[a]);
  invariant eq_pc   : (cpu_i_1->pc == cpu_i_2->pc);
  invariant eq_inst : (cpu_i_1->inst == cpu_i_2->inst);

  // Proof script.
  control {
    unroll(3);
    check;
    print_results;
  }
}
\end{lstlisting}
\caption{Module \codelike{main} in the CPU model}
\label{ex:cpu-main}
\end{uclidlisting}


Isolating commonly used types into a single module in this manner allows the construction of large models parameterized by this types. These common types can be changed and the ramifications of these changes on the model's behavior can be studied easily.

\section{Uninterpreted Functions and Types}
A convenient mechanism for abstraction in \uclid{} is through the use of uninterpreted functions and types.

\subsection{Uninterpreted Types}
Example~\ref{ex:cpu-cpu-p1} shows the use of the \emph{uninterpreted type}: \codelike{regindex_t} on line~9. The index type to the register is an \emph{abstract} type, as opposed to a specific type (e.g. \codelike{bv3}). This allows us to reason about an abstract register file that has an undefined (and unbounded) number of entries, as opposed to proving facts about some specific register file implementation, potentially enabling more general proofs about system behavior.

\subsection{Uninterpreted Functions}
Values belonging to an uninterpreted type can be created using \emph{uninterpreted functions}. The functions \codelike{inst2op}, \codelike{inst2reg0}, \codelike{inst2reg1}, \codelike{inst2imm} and \codelike{inst2addr} on lines~20-24 of Example~\ref{ex:cpu-cpu-p1} are all examples of uninterpreted functions. An uninterpreted function $f$ is a function about which we know nothing, except that it is a function; i.e. $\forall x_1, x_2.~ x_1 = x_2 \implies f(x_1) = f(x_2)$.

Uninterpreted functions allow us to reason about an abstract CPU model without considering specific instruction encodings or decoder models. This could potentially lead to more general proofs as well as more scalable automated proofs. 


\section{Procedure Definition}
Besides modules, another way of managing model complexity in \uclid{} is through the use of \emph{procedures}. Example~\ref{ex:cpu-cpu-p1} defines the \keyword{procedure} \codelike{exec_inst} on line~29. The procedure takes two arguments: \codelike{inst} and \codelike{pc} and returns two values: \codelike{result} and \codelike{pc_next}.

Note that all module-level state modified by the procedure must be declared using a \keyword{modifies} clause; this is shown on line~31.

\subsection{Procedure Invocation}
Procedures are invoked using the \keyword{call} keyword. In Example~\ref{ex:cpu-cpu-p2}, procedure \codelike{exec_inst} is called on line~15 of Example~\ref{ex:cpu-cpu-p2}. The actual arguments to the procedure are \codelike{inst} and \codelike{pc}, and the return values of the procedure will be stored in the variables \codelike{result} and \codelike{pc}.

\section{Module Instantiation and Scheduling}

Modules are instantiated using the \keyword{instance} keyword. Lines 14~and~15 of Example~\ref{ex:cpu-main} show two instantiations of the module \codelike{cpu}. For each instance, the module input \codelike{imem} is mapped to the state variable \codelike{imem} of module \codelike{main}.

Scheduling of instantiated modules is explicit and synchronous. The two \keyword{next} statements on lines 22~and~23 of Example~\ref{ex:cpu-main} invoke the next state transitions of the two instances of the \codelike{cpu} module.

\subsection{Accessing Instance Variables}
The state variables of an instantiated module are accessed using the \codelike{->} operator. The four invariants on lines 27-32 of Example~\ref{ex:cpu-main}, refer to the registers, memory, pc and instruction variables of the two instantiated modules. These invariants state that both instances must have identical values for these state variables.


\section{Running \uclid{}}

Executing \uclid{} on the complete CPU model shows that CPU is in fact deterministic. 

\begin{Verbatim}[frame=single, samepage=true]
$ uclid examples/tutorial/ex4.1-cpu.ucl 
16 assertions passed.
0 assertions failed.
0 assertions indeterminate.
\end{Verbatim}

\subsection{Exercise: Inductive Proof of CPU model}
Prove determinism of the CPU model using induction rather than bounded model checking. You will need to add strengthening inductive invariants relating the two CPU instances.
