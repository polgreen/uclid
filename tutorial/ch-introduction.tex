\chapter{Introduction}

\uclid{} is a software toolkit for the formal modeling, specification,
verification, and synthesis of computational systems. 
The \uclid{} toolchain aims to:
\begin{enumerate}
\item Enable compositional (modular) modeling of finite and infinite state transition systems across a range of concurrency models and background logical theories;
\item Verification of a range of properties, including assertions, invariants, and temporal properties, and % mention hyperproperties in a future release
\item Integrate modeling and verification with algorithmic and inductive synthesis.
\end{enumerate}

\uclid draws inspiration from the earlier {UCLID} system for modeling
and verification of systems~\cite{bryant-cav02,uclid-www}, in particular
the idea of modeling concurrent systems in first-order logic 
with a range of background theories, and the use of proof scripts
within the model. However, the \uclid modeling
language and verification capabilities go beyond the original modeling 
language, and the planned integration with synthesis is novel.

This document serves as introduction to the \uclid{} modeling 
language and toolchain. With the \uclid system under active
development, we expect this document to undergo several changes
as the system and its applications evolve.

\section{Getting Started: A Simple \uclid{} Model}

\begin{uclidlisting}[htbp]
    \lstinputlisting[language=uclid,style=uclidstyle]{../examples/tutorial/ex1.1-fib-model.ucl}
    \label{ex:fib-model}
\caption{A \uclid{} model that computes the Fibonacci sequence}
\end{uclidlisting}

A simple \uclid{} module that computes the Fibonacci sequence is shown in Example~\ref{ex:fib-model}.  We will now walk through each line in this model to understand the basics of \uclid{}.

The top-level syntactic structure in \uclid{} is a \keywordbf{module}. All modeling, verification and synthesis code in \uclid{} is contained within modules. In Example~\ref{ex:fib-model}, we have defined one \keyword{module} named \ident{main}. This module starts on line 1 and ends on line 18. The module can be conceptually split into three parts: a system model, a specification and proof script. 

In the example, these three conceptual parts are also kept separate in the code.\footnote{This is not required by \uclid{} syntax, as \keywordbf{invariant} declarations and assumptions can be interleaved with \keyword{init}, \keyword{next}, \keyword{var} declarations as well other types of declarations. However, keeping these conceptually different parts separate is good design practice. \uclid{} does require that if a \keyword{control} block is specified, then it is the very last element of a module.} The following subsections will describe each of these sections of the module. 

\subsubsection{The System Model}
This part of a \uclid{} module describes the functionality of the transition system that is being modeled: it tells us \emph{what the system does}.

The first item of interest within the module \ident{main} are \emph{state variables}. These are declared using the \keywordbf{var} keyword. The module \ident{main} declares two state variables: \ident{a} and \ident{b} on line~3. These are both of type \keywordbf{integer}, which corresponds to mathematical integers.\footnote{Mathematical integer types, as opposed to the machine integer types present in languages like C/C++ and Java, do not have a fixed bit-width and do not overflow.}

The \keywordbf{init} block appears next and spans lines 5 to 8. It defines the initial values of the state variables in the module. The notation \codelike{a'} refers to the value of the state variable \ident{a} at the end of the current ``step'', which in this case refers to initial state. The model is specifying that after the \codelike{init} block is executed, \codelike{a} and \codelike{b} have the values 0 and 1 respectively.

The \keywordbf{next} block appears after this and it defines the transition relation of the module. In the figure, the next statement spans from lines 9 to 11; \ident{a} is assigned to the (old) value of \ident{b}, while b is assigned to the value \ident{a} + \ident{b}.

\subsubsection{The System Specification}
The specification answers the question: \emph{what is the system supposed to do?}. 

In our example, we have a single \keywordbf{invariant} that comprises that entire specification. Line 14 defines this \keyword{invariant}. It is named \ident{a\_le\_b} and as the name suggests, it states that \ident{a} must be less than or equal to \ident{b} for every reachable state of the system.

\subsubsection{The Proof Script}
The third and final part of the \uclid{} module is a set of commands to the \uclid{} verification engine. These tell how we should go about proving\footnote{We are using a broad definition of the word ``prove'' here to refer to any systematic method that gives us assurance that the specification is satisfied.} that the system satisfies its specification.

The proof script is contained within the \keywordbf{control} block. The commands here execute the system for 3 steps and check whether all of the systems properties (in this case, we only have one invariant: \ident{a\_le\_b}) are satisfied for each of these steps. 

The command \proofcmd{unroll} executes the system for 3 steps. This execution generates four \emph{proof obligations}. These proof obligations ask whether the system satisfies the invariant \ident{a\_le\_b} in the initial state and in each of the 3 states reached next. The \proofcmd{check} command \emph{checks} whether these proof obligations are satisfied and the \proofcmd{print\_results} prints out the results of these checks.

\section{Installing \uclid{}}

Public releases of the \uclid{} can be obtained at: \url{https://github.com/uclid-org/uclid/releases}. For the impatient, the short version of the installation instructions is: download the archive with the latest release, unzip the archive and add the `bin/' subdirectory to your \codelike{PATH}. 

More detailed instructions for installation are as follows.

\subsection{Prerequisites}

\uclid{} has two prerequisites. 
\begin{enumerate}
    \item \uclid{} requires that the Java\texttrademark{} Runtime Environment be installed on your machine. You can download the latest Java Runtime Environment for your platform from \url{https://www.java.com}. 
\item \uclid{} uses the Z3 SMT solver. You can install Z3 from: \url{https://github.com/Z3Prover/z3/releases}. Make sure the `z3' or `z3.exe' binary is in your path after Z3 installed. Also make sure, the shared libraries for libz3 and libz3java are in the dynamic library load path (\codelike{LD\_LIBRARY\_PATH} on Unix-like systems).
\end{enumerate}

\uclid{} has been tested with Java\texttrademark{} SE Runtime Environment version 1.8.0 and Z3 versions 4.5.1 and 4.6.0.

\subsection{Detailed Installation Instructions}

First, down the platform independent package from \url{https://github.com/uclid-org/uclid/releases}.

Next, follow these instructions which are provided for the bash shell running on a Unix-like platform. Operations for Micosoft Windows, or a different shell should be similar.

\begin{itemize}
  \item Unzip the archive.

  \codelike{\$ unzip uclid-0.9.5.zip}.

\begin{comment}
  This should produce output similar to the following.
\begin{Verbatim}[frame=single]
Archive:  uclid-0.9.5.zip
  inflating: uclid-0.9.5/lib/default.uclid-0.9.5.jar  
  inflating: uclid-0.9.5/lib/com.microsoft.z3.jar  
  inflating: uclid-0.9.5/lib/org.scala-lang.scala-library-2.12.0.jar  
  inflating: uclid-0.9.5/lib/org.scala-lang.modules.scala-parser-combinators_2.12-1.0.6.jar  
  inflating: uclid-0.9.5/lib/org.scalactic.scalactic_2.12-3.0.1.jar  
  inflating: uclid-0.9.5/lib/org.scala-lang.scala-reflect-2.12.0.jar  
  inflating: uclid-0.9.5/bin/uclid     
  inflating: uclid-0.9.5/bin/uclid.bat  
\end{Verbatim}
\end{comment}


  \item Add the \codelike{uclid} binary to your path.

  \codelike{\$ export PATH=\$PATH:\$PWD/uclid-0.9.5/bin/}

  \item Check that the \texttt{uclid} works.

  \codelike{\$ uclid}

  This should produce output similar to the following.
\begin{Verbatim}[frame=single, samepage=true]
$ uclid

    Usage: uclid [options] filename [filenames]
    Options:
      -h/--help : This message.
      -m/--main : Set the main module.
      -d/--debug : Debug options.
  
Error : Unable to find main module.
\end{Verbatim}
\end{itemize}

\subsection{Running \uclid{}}

Invoke \uclid{} on a model is easy. Just run the \codelike{uclid} binary and provide a list of files containing the model as a command-line argument. When invoked, \uclid{} will parse each of these files and look for a module named \codelike{main} among them. It will execute the commands in the \codelike{main} module's control block. The \codelike{--main} command line argument can be used to specify a different name for the ``main'' module. Note only the \codelike{main} module's control blocks will be executed, even if the \codelike{main} module instantiates other modules with control blocks. If no \codelike{main} module is found, \uclid{} will exit with an error, as we saw in the previous section when \codelike{uclid} was invoked without arguments.

Example~\ref{ex:fib-model} is part of the \uclid{} distribution in the \codelike{examples/tutorial/} sub-directory. You can run \uclid{} on this model as:

\begin{verbatim}
$ uclid examples/tutorial/ex1.1-fib-model.ucl 
\end{verbatim}

This should produce the following output.
\begin{Verbatim}[frame=single, samepage=true]
Successfully parsed 1 and instantiated 1 module(s).
4 assertions passed.
0 assertions failed.
0 assertions indeterminate.
Finished execution for module: main.
\end{Verbatim}

\section{Looking Forward}

This chapter has provided an brief overview of \uclid{}'s features and toolchain. The rest of this tutorial will take a more detailed looked at more of \uclid{}'s features.
