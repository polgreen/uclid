\chapter{Verification Techniques}
\label{ch:verif}

In the examples covered thus far, we have only used \uclid{} for bounded model checking of invariants. 
\uclid{} can also be used to do unbounded inductive proofs and also provides support for debugging counterexamples. This chapter will describe these features of \uclid{}. Further features are being implemented and will be described in a future version of this document.

\section{Inductive Proofs} % change to \section when we have more sections!
Let us revisit the model from Example~\ref{ex:fib-model}. This is now shown again in Example~\ref{ex:fib-induction}, but with a different proof script. Instead of using the \codelike{unroll} command for bounded model checking, we are using the \codelike{induction} command to attempt an inductive proof.

\begin{uclidlisting}[htbp]
    \lstinputlisting[language=uclid,style=uclidstyle]{../examples/tutorial/ex3.1-fib-induction.ucl}
    \label{ex:fib-induction}
    \caption{\uclid{} Fibonacci model using induction in the proof script}
\end{uclidlisting}

\subsection{Debugging Counterexamples}

Let us try running \uclid{} on Example~\ref{ex:fib-induction} with the new proof script.
\begin{Verbatim}[frame=single, samepage=true]
$ uclid examples/tutorial/ex3.1-fib-induction.ucl 
Successfully parsed 1 and instantiated 1 module(s).
1 assertions passed.
1 assertions failed.
0 assertions indeterminate.
  FAILED -> induction (step) [Step #1] 
  property a_le_b @ ex3.1-fib-induction.ucl, line 14
Finished execution for module: main.
\end{Verbatim}

Uh oh, we seem to have a problem! \uclid{} is telling us that the inductive proof failed. We can try to examine why the proof failed by using the \codelike{print_cex} command to examine the counterexample to the proof.

\begin{uclidlisting}[htbp]
    \lstinputlisting[language=uclid,style=uclidstyle]{../examples/tutorial/ex3.2-fib-induction-cex.ucl}
    \caption{\uclid{} Fibonacci model with \codelike{induction} and \codelike{print_cex}}
    \label{ex:fib-induction-cex}
\end{uclidlisting}

The only changes between Example~\ref{ex:fib-induction} and Example~\ref{ex:fib-induction-cex} are on lines~18 and 21. \codelike{vobj} on line~18 is a reference to the verification conditions generated by the \codelike{induction} command. On line~21, we pass this reference to the \codelike{print_cex} command which prints out the values of \codelike{a} and \codelike{b} for the counterexample.

Running \uclid{} on Example~\ref{ex:fib-induction-cex} produces the following.

\begin{Verbatim}[frame=single, samepage=true]
Successfully parsed 1 and instantiated 1 module(s).
1 assertions passed.
1 assertions failed.
0 assertions indeterminate.
  FAILED -> vobj: induction (step) [Step #1] 
  property a_le_b @ ex3.2-fib-induction-cex.ucl, line 14
CEX for vobj: induction (step) [Step #1] 
property a_le_b @ ex3.2-fib-induction-cex.ucl, line 14
=================================
Step #0
  a : -1
  b : 0
=================================
=================================
Step #1
  a : 0
  b : -1
=================================
Finished execution for module: main.
\end{Verbatim}

To understand the counterexample, it is helpful to review how the inductive proof engine works. When inductively proving the \keyword{invariant} \codelike{a_le_b}, \uclid{} considers some arbitrary state that satisfies this property, executes the \keyword{next} block, and checks whether \codelike{a_le_b} holds on the resultant state.

The counterexample shows us that we do start in a state where $\codelike{a} \le \codelike{b}$ with $\codelike{a}=-1$ and $\codelike{b}=0$. We execute the \keyword{next} block and now \codelike{a} gets the value of \codelike{b}, becoming 0 and \codelike{b} gets the value $\codelike{a} + \codelike{b}$, becoming -1. This new state does not satisfy the invariant!

What is the real problem here? Taking a closer look at Example~\ref{ex:fib-induction-cex}, we see that this specific counterexample can never occur in our model because \codelike{a} and \codelike{b} are always $\ge 0$. But \uclid{} does not know this when attempting the inductive proof. Therefore, we have to strengthen the inductive argument with this information in order to help \uclid{}'s proof.

\subsection{Inductive Proof for the Fibonacci Model}

\begin{uclidlisting}[htbp]
    \lstinputlisting[language=uclid,style=uclidstyle]{../examples/tutorial/ex3.3-fib-induction-proof.ucl}
    \caption{Inductive proof for the Fibonacci model}
    \label{ex:fib-induction-proof}
\end{uclidlisting}

Example~\ref{ex:fib-induction-proof} shows the same model as Example~\ref{ex:fib-induction-cex}, but with a stronger induction hypothesis. \uclid{}'s inductive engine will now start in an arbitrary state that assumes that both invariants \codelike{a_le_b} and \codelike{a_b_ge_0} hold and attempt to prove that both of these still hold after the \keyword{next} block is executed.

Let us now run \uclid{} on this new model.

\begin{Verbatim}[frame=single, samepage=true]
Successfully parsed 1 and instantiated 1 module(s).
$ uclid examples/tutorial/ex3.3-fib-induction-proof.ucl 
4 assertions passed.
0 assertions failed.
0 assertions indeterminate.
Finished execution for module: main.
\end{Verbatim}

Success! We have shown that our system model satisfies its specification.

\section{Bounded Model Checking}

\begin{uclidlisting}[htbp]
    \lstinputlisting[language=uclid,style=uclidstyle]{../examples/tutorial/ex3.4-fib-model-revisited.ucl}
    \caption{Revisiting the Fibonacci model from Example~\ref{ex:fib-model}.}
    \label{ex:fib-model-v2}
\end{uclidlisting}

Let us return to the model of Example~\ref{ex:fib-model} which is reproduced as Example~\ref{ex:fib-model-v2} with a few changes. We used the \codelike{unroll} command for verification. This command performs bounded model checking and takes a single argument -- the number of steps to unroll the model for. In Example~\ref{ex:fib-model-v2}, we are unrolling the model for 3 steps. We have introduced the constant \codelike{flag} on line~4. A constant holds a symbolic value that does not change during computation. The initial value of the constant is assigned non-deterministically and can be controlled using assumptions.

\subsection{Embedded assume and assert statements}

A second difference with between Example~\ref{ex:fib-model} and Example~\ref{ex:fib-model-v2} is on lines 12--14, 24 and 25.  Instead of using a module-level assumption declarations as in Example~\ref{ex:fib-model}, we have three embedded assumptions in the \codelike{set\_init} procedure on lines 12--14, and two embedded assertions in the \keywordbf{next} block on lines 23 and 25. A module-level assumption is assumed to hold solver at every step of execution, while an embedded assumption is assumed ``instantaneously.'' In particular, the assumptions on lines~12--14 tells the solver to assume that $\codelike{a} \leq \codelike{b}$, $\codelike{a} >= 0$ and $\codelike{b} >= 0$ at the end of the \keywordbf{set\_init} procedure. Notice that we are not assigning specific values to \codelike{a} and \codelike{b}, instead we are asking \uclid{} to consider potential values of \codelike{a} and \codelike{b} such that $\codelike{a} \leq \codelike{b}$, $\codelike{a} \geq 0$ and $\codelike{b} \geq 0$.

Similarly the assertions on lines 23 and 25 are evaluated at that specific location in the code. In particular the assertion on line 23 is only checked when \codelike{flag} is \codelike{true}, while the assertion one line 25 is checked when \codelike{flag} is \codelike{false}. Since \codelike{flag} is always \codelike{true} in our model, the assertion on line 25 will never fire. In contrast, note that a module-level assertion would be evaluated after the \keywordbf{init} block and after each execution of the \keywordbf{next} block.

\subsection{Running \uclid{}}

Running \uclid{} on Example~\ref{ex:fib-model-v2} shows that the embedded assertions do indeed hold for all states reachable within 3 steps of the initial state.

\begin{Verbatim}[frame=single, samepage=true]
$ uclid examples/tutorial/ex3.4-fib-model-revisted.ucl 
Successfully parsed 1 and instantiated 1 module(s).
6 assertions passed.
0 assertions failed.
0 assertions indeterminate.
Finished execution for module: main.
\end{Verbatim}

\section{Specifications in Linear Temporal Logic}
\begin{uclidlisting}[htbp]
    \lstinputlisting[language=uclid,style=uclidstyle]{../examples/tutorial/ex3.5-traffic-light-ltl.ucl}
    \caption{Example of using LTL specifications in \uclid{}.}
    \label{ex:traffic-light-ltl}
\end{uclidlisting}

\uclid{} supports the specification of module behavior using linear temporal logic (LTL). 

Example~\ref{ex:traffic-light-ltl} shows a \uclid{} model of an intersection with two traffic lights. Lines~3--39 define the functionality of the traffic light; this part of the model should be familiar. The current state of the lights are stored in the variables \codelike{light1} and \codelike{light2}, and these switch from \codelike{red} to \codelike{green} to \codelike{yellow} and back to \codelike{red}. The variables \codelike{step1} and \codelike{step2} can be thought of timers, and ensure that each light stays red for three transitions, green for two transitions and stays yellow for a single transition. 

The LTL properties are on lines 41, 42 and 43. The property \codelike{always\_one\_red} specifies a safety property which states that at least one of the two lights must be \codelike{red} in every particular cycle. The notation $\codelike{G}(\phi)$ refers to the LTL globally operator, while the notation $\codelike{F}(\phi)$ refers to the LTL eventually (future) operator. Other supported operators include next-time: $\codelike{X}(\phi)$, (strong-)until: $\codelike{U}(\phi_1, \phi_2)$ and weak-until: $\codelike{W}(\phi_1, \phi_2)$. \codelike{always\_one\_red} is safety property. The property \codelike{eventually\_green} is an example of liveness property, and specifies that both lights become \codelike{green} infinitely often.

The command for bounded verification of LTL properties is the \codelike{bmc} command. This is invoked on line 46 and specifies which properties must be checked within the square brackets. (If no properties are specified, and the square brackets are omitted \codelike{bmc} checks all LTL properties in the module.)

\subsection{Running \uclid{}}
Running \uclid{} on Example~\ref{ex:traffic-light-ltl} produces the following output.
\begin{Verbatim}[frame=single, samepage=true]
$ uclid run examples/traffic-light.ucl
Running (fork) uclid.UclidMain examples/traffic-light.ucl
Successfully parsed 1 and instantiated 1 module(s).
44 assertions passed.
0 assertions failed.
0 assertions indeterminate.
Finished execution for module: main.
\end{Verbatim}

The output shows that all properties are verified.

\subsubsection{Exercises}
\begin{enumerate}
    \item Does the property \codelike{always\_one\_red} hold if the assignment to \codelike{step2} on line~9 is changed to 2 (from 1)? Why or why not? Make this change, print-out and understand the counterexample if one exists.

    \item Find a way to modify the model so that the property \codelike{eventually\_green} is violated. Examine and understand the counter-example generated by \uclid{} when this happens.
\end{enumerate}
\section{Future Directions}

Future versions of \uclid{} will have support for synthesizing invariants using Syntax-Guided Synthesis (SyGuS).
