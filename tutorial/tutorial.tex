\documentclass{article}
\usepackage{syntax}
\usepackage{color}
\usepackage{booktabs}
\usepackage{multirow}
\usepackage{multicol}
\usepackage{amsmath}
\usepackage{xspace}
\usepackage{pgf}
\usepackage{tikz}
\usepackage{comment}
\usepackage{xifthen}
\usepackage{listings}

\makeatletter
\makeatother

\usetikzlibrary{shapes}
\usetikzlibrary{snakes}
\usetikzlibrary{patterns}
\usetikzlibrary{circuits}
\usetikzlibrary{arrows}
\usetikzlibrary{calc}
\usetikzlibrary{fit}

\let\ab\allowbreak

\definecolor{codegreen}{rgb}{0,0.6,0}
\definecolor{codegray}{rgb}{0.5,0.5,0.5}
\definecolor{codepurple}{rgb}{0.58,0,0.82}
\definecolor{backcolour}{rgb}{0.95,0.95,0.92}
 
\newcommand{\uclid}{{\sc{Uclid}}}
\newcommand{\keyword}[1]{\textcolor{magenta}{\texttt{#1}}}
\newcommand{\ident}[1]{\textcolor{blue}{\texttt{#1}}}
\newcommand{\keywordbf}[1]{\textcolor{magenta}{\texttt{\textbf{#1}}}}

\lstdefinestyle{uclid}{
    backgroundcolor=\color{backcolour},   
    commentstyle=\color{codegreen},
    keywordstyle=\color{magenta},
    numberstyle=\tiny\color{codegray},
    stringstyle=\color{codepurple},
    basicstyle=\footnotesize,
    breakatwhitespace=false,         
    breaklines=true,                 
    captionpos=b,                    
    keepspaces=true,                 
    numbers=left,                    
    numbersep=5pt,                  
    showspaces=false,                
    showstringspaces=false,
    showtabs=false,                  
    tabsize=2
}

\lstdefinelanguage{uclid}{
  sensitive = true,
  keywords={module, forall, exists, Lambda, if, else, assert, assume, havoc, 
            for, skip, case, esac, init, next, control, function, procedure, 
            returns, call, type, var, input, output, const, property, 
            invariant, synthesis, grammar, requires, ensures, instance, axiom},
 %otherkeywords={% Operators
 %   >, <, ==, <=, >=, +, -, *, !=
 % },
  numbers=left,
  numberstyle=\scriptsize,
  stepnumber=1,
  numbersep=8pt,
  showstringspaces=false,
  breaklines=true,
  frame=top,
  comment=[l]{//},
  morecomment=[s]{/*}{*/},
}

\definecolor{codegreen}{rgb}{0,0.6,0}
\definecolor{codegray}{rgb}{0.5,0.5,0.5}
\definecolor{codepurple}{rgb}{0.58,0,0.82}
\definecolor{backcolour}{rgb}{1.0,1.0,1.0}
 
\lstdefinestyle{uclidstyle}{
    backgroundcolor=\color{backcolour},   
    commentstyle=\color{codegreen},
    keywordstyle=\color{magenta},
    numberstyle=\tiny\color{codegray},
    stringstyle=\color{codepurple},
    basicstyle=\footnotesize,
    breakatwhitespace=false,         
    breaklines=true,                 
    captionpos=b,                    
    keepspaces=true,                 
    numbers=left,                    
    numbersep=5pt,                  
    showspaces=false,                
    showstringspaces=false,
    showtabs=false,                  
    tabsize=2
}

\begin{document}
\title{Getting Started with \uclid{}}
\maketitle

\section{Introduction}
\uclid{} is a verification and synthesis focused modeling language. The \uclid{} toolchain aims to:
\begin{enumerate}
\item Enable modeling of finite and infinite state transition systems.
\item Verification of safety and k-safety properties on these systems. 
\item Allow syntax-guided synthesis of models and model invariants on these transitions systems.
\end{enumerate}

This document serves as introduction to \uclid{} modeling language and verification/synthesis toolchain.

\subsection{A Simple \uclid{} Model}

\begin{figure}[htbp]
\begin{lstlisting}[language=uclid,style=uclidstyle]
module main {
  var a, b : int;

  init {
    a = 0b0; // binary literal
    b = 0x1; // hexadecimal literal
  }
  next {
    a, b = b, a + b;
  }

  property a_le_b: a <= b;
  
  control {
    unroll (3);
    decide;
  }
}
\end{lstlisting}
\label{fig:simple-model}
\caption{A \uclid{} model that computes the Fibonacci sequence.}
\end{figure}

A simple \uclid{} module that computes the Fibonacci sequence is shown in Figure~\ref{fig:simple-model}.  Let us walk through each line in this model to understand the basics of \uclid{}.

The top-level syntactic structure in \uclid{} is a \keywordbf{module}. All modeling, verification and synthesis code in \uclid{} is contained within modules. In Figure~\ref{fig:simple-model}, we have defined one \keyword{module} named \ident{main}. This module starts on line 1 and ends on line 18.

The next item of interest in the module \ident{main} are \emph{state variables}. These are declared using the \keywordbf{var} keyword. The module \ident{main} declares two state variables: \ident{a} and \ident{b} on line~2. These are both of type \keywordbf{int}, which corresponds to mathematical integers.\footnote{Mathematical integer types, as opposed to the machine integer types present in languages like C/C++ and Java, do not have a fixed bit-width and do not overflow.}

The \keywordbf{init} block appears next and spans lines 4 to 7. It defines the initial values of the states variables in the module. We see that \ident{a} is initialized to 0 while \ident{b} is initialized to 1.

The \keywordbf{next} block appears after this and it defines the transition relation of the module. In the figure, the next statement spans from lines 8 to 10; \ident{a} is assigned to the (old) value of \ident{b}, while b is assigned to the value \ident{a} + \ident{b}. 

\paragraph{Default settings}
\begin{grammar}

<statement> ::= <ident> `=' <expr> 
\alt `for' <ident> `=' <expr> `to' <expr> `do' <statement> 
\alt `{' <stat-list> `}' 
\alt <empty> 

<stat-list> ::= <statement> `;' <stat-list> | <statement> 

\end{grammar}
\paragraph{Increase the two lengths}
\setlength{\grammarparsep}{20pt plus 1pt minus 1pt} % increase separation between rules
\setlength{\grammarindent}{12em} % increase separation between LHS/RHS 

\begin{grammar}

<statement> ::= <ident> `=' <expr> 
\alt `for' <ident> `=' <expr> `to' <expr> `do' <statement> 
\alt `{' <stat-list> `}' 
\alt <empty> 

<stat-list> ::= <statement> `;' <stat-list> | <statement> 

\end{grammar}
\end{document}
